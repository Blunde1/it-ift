% Appendix C

\chapter{Code Snippets}
\label{AppendixC}
\lhead{Appendix C. \emph{Code Snippets}}

\section{Additions to the \textbf{TMB} Package}

For practical purposes (cf. section \ref{Chap5.3.1}), the modified Bessel function of the first kind and the log-normal density were made available to the \textbf{TMB} user.
Since R is running in the background and R is based on C++, the function values can be drawn from R since both functions are already implemented in R.
The partial derivatives of the function must then be implemented manually.
For these purposes, the first code snippet \eqref{lst:atomic_math} was placed inside the file "atomic\_math.hpp" in the "include" folder of the \textbf{TMB} package.
In addition, to ease user implementation, the second code \eqref{lst:convenience} was placed inside the file "convenience.hpp" of the same folder.
We note that for the modified Bessel function of the first kind, a finite difference approximation of the derivative with respect to $\nu$ was used, due to the complicated expression of this term.
This goes against the exactness of AD, but was tested and found to work well in practice.

\begin{frame}
	
	\lstset{
		language=C++,
		basicstyle=\small,
		belowcaptionskip=1\baselineskip,
		breaklines=true,
		basicstyle=\footnotesize,
		backgroundcolor=\color{white!5}, % set backgroundcolor
		keywordstyle=\color{red}\ttfamily,
		otherkeywords={Type},
		stringstyle=\color{orange}\ttfamily,
		commentstyle=\color{blue}\ttfamily,
		morecomment=[l][\color{purple}]{\#},
		frame=single,
		%numbers=left,
		numbersep=12pt,
		numberstyle=\tiny\color{blue}, % the style that is used for the line-numbers
		rulecolor=\color{black},
		frameround=tttt
	}
	
	\begin{lstlisting}[caption={Addition to atomic\_math.hpp},label={lst:atomic_math},]
	TMB_ATOMIC_VECTOR_FUNCTION(
	// ATOMIC_NAME
	besselI
	,
	// OUTPUT_DIM
	1
	,
	// ATOMIC_DOUBLE
	ty[0] = Rmath::Rf_bessel_i(tx[0], tx[1], 1.0 /* Not scaled */ );
	,
	// ATOMIC REVERSE
	Type value = ty[0];
	Type x = tx[0];
	Type nu = tx[1];
	CppAD::vector<Type> arg(2);
	arg[0] = x;
	arg[1] = nu + Type(1);
	px[0] = (besselI(arg)[0] + value * (nu / x)) * py[0];
	arg[1] = nu + Type(0.000001);
	px[1] = ((besselI(arg)[0] - besselI(tx)[0]) / Type(0.0001))* py[0];
	)
	
	TMB_ATOMIC_VECTOR_FUNCTION(
	// ATOMIC_NAME
	dlnorm
	,
	// OUTPUT_DIM
	1
	,
	// ATOMIC_DOUBLE
	ty[0] = Rmath::Rf_dlnorm(tx[0],tx[1],tx[2],0 /* log=FALSE */ ); 
	,
	// ATOMIC_REVERSE
	px[0] = -ty[0]/tx[0] * (1-(log(tx[0])-tx[1]) / (tx[2]*tx[2])) * py[0];
	px[1] = ty[0] * (log(tx[0])-tx[1]) / (tx[2]*tx[2]) * py[0];
	px[2] = -ty[0] / tx[2] * (1 - ((log(tx[0])-tx[1]) * 
			(log(tx[0])-tx[1]) / (tx[2]*tx[2]))) * py[0];
	)
	
	\end{lstlisting}

	\begin{lstlisting}[caption={Addition to convenience.hpp},label={lst:convenience}]
	template<class Type>
	Type besselI(Type x, Type nu) {
	CppAD::vector<Type> tx(2);
	tx[0] = x;
	tx[1] = nu;
	return atomic::besselI(tx)[0];
	}
	
	template<class Type>
	Type dlnorm(Type x, Type mu, Type sigma){
	CppAD::vector<Type> tx(3);
	tx[0] = x;
	tx[1] = mu;
	tx[2] = sigma;
	return atomic::dlnorm(tx)[0];
	}
	\end{lstlisting}
	
\end{frame}

\begin{frame}
	
	\lstset{
		language=C++,
		basicstyle=\small,
		belowcaptionskip=1\baselineskip,
		breaklines=true,
		basicstyle=\footnotesize,
		backgroundcolor=\color{white!5}, % set backgroundcolor
		keywordstyle=\color{red}\ttfamily,
		%otherkeywords={Type},
		morekeywords={cType, Type},
		stringstyle=\color{orange}\ttfamily,
		commentstyle=\color{blue}\ttfamily,
		morecomment=[l][\color{purple}]{\#},
		frame=single,
		%numbers=left,
		numbersep=12pt,
		numberstyle=\tiny\color{blue}, % the style that is used for the line-numbers
		rulecolor=\color{black},
		frameround=tttt
	}
	

	
	\begin{lstlisting}[caption={Creation of the complex cType data type in TMB},label={lst:complex}]
/**
* Makes the following available for the TMB user:
* The "cType<Type>" complex AD data type. 
* The arithmetic follows standard arithmetic for complex variables.
* It is defined to work together with the standard TMB data type "Type".
* Constructor: cType<Type> z; defaults to 0+0*i.
*              cType<Type> z((Type)Re,(Type)Im) = Re + i*Im.
* Compound assignements (+=, -=, *=, /=).
* Relational and comparison operators (==, !=).
* Standard functions for complex variables (abs, arg, conj).
* Exponential functions (exp, log).
* Power functions (pow, sqrt).
* Trigonometric functions (sin, cos, tan, asin, acos, atan, sinh, cosh, tanh).
* 
*/

template<class Type>
struct cType{

// MEMBER FUNCTIONS
Type r, i;

// Constructor
cType(void) {r=0;i=0;}
cType(Type r_, Type i_) : r(r_), i(i_) {}

// Compound assignements
cType& operator =(const cType& c){
r = c.r;
i = c.i;
return *this;
}
cType& operator +=(const Type& t){
r = r + t;
return *this;
}
cType& operator +=(const cType& c){
r = r + c.r;
i = i + c.i;
return *this;
}
cType& operator -=(const Type& t){
r = r - t;
return *this;
}
cType& operator -=(const cType& c){
r = r - c.r;
i = i - c.i;
return *this;
}
cType& operator *=(const Type t){
r = r*t;
i = i*t;
return *this;
}
cType& operator *=(const cType c){
Type tmp_r, tmp_i;
tmp_r = r*c.r - i*c.i;
tmp_i = r*c.i + i*c.r;
r = tmp_r, i=tmp_i;
return *this;
}
cType& operator /=(const Type t){
r = r / t;
i = i / t;
return *this;
}
cType& operator /=(const cType c){
Type div = c.r*c.r + c.i*c.i, tmp_r, tmp_i;
tmp_r = (r*c.r + i*c.i)/div;
tmp_i = (i*c.r - r*c.i)/div;
r = tmp_r, i = tmp_i;
return *this;
}
};

// NON-MEMBER FUNCTIONS

// Arithmetic
template<class Type>
cType<Type> operator +(const cType<Type>& c, const Type& t){
cType<Type> res = c;
return res+=t;
}
template<class Type>
cType<Type> operator +(const Type& t, const cType<Type>& c){
cType<Type> res = c;
return res+=t;
}
template<class Type>
cType<Type> operator +(const cType<Type>& c_1, const cType<Type>& c_2){
cType<Type> res = c_1;
return res += c_2;
}
template<class Type>
cType<Type> operator -(const cType<Type>& c, const Type& t){
cType<Type> res = c;
return res-=t;
}
template<class Type>
cType<Type> operator -(const Type& t, const cType<Type>& c){
cType<Type> res(t,0);
return res -= c;
}
template<class Type>
cType<Type> operator -(const cType<Type>& c_1, const cType<Type>& c_2){
cType<Type> res = c_1;
return res -= c_2;
}
template<class Type>
cType<Type> operator *(const cType<Type>& c, const Type& t){
cType<Type> res = c;
return res *= t;
}
template<class Type>
cType<Type> operator *(const Type& t, const cType<Type>& c){
cType<Type> res(t,0);
return res *= c;
}
template<class Type>
cType<Type> operator *(const cType<Type>& c_1, const cType<Type>& c_2){
cType<Type> c1 = c_1, c2=c_2;
return c1 *= c2;
}
template<class Type>
cType<Type> operator /(const cType<Type>& c, const Type& t){
cType<Type> res = c;
return res /= t;
}
template<class Type>
cType<Type> operator /(const Type& t, const cType<Type>& c){
cType<Type> res(t,0);
return res /= c;
}
template<class Type>
cType<Type> operator /(const cType<Type>& c_1, const cType<Type>& c_2){
cType<Type> res = c_1;
return res /= c_2;
}

// Relational and comparison operators
template<class Type>
bool operator ==(const cType<Type>& lhs, const cType<Type>& rhs){
cType<Type> c_1=lhs, c_2 = rhs;
if(c_1.r == c_2.r && c_1.i == c_2.i) {return true;}
else{return false;}
}
template<class Type>
bool operator ==(const cType<Type>& lhs, const Type& rhs){
cType<Type> c_1=lhs, c_2(rhs,0);
if(c_1.r == c_2.r && c_1.i == c_2.i) {return true;}
else{return false;}
}
template<class Type>
bool operator ==(const Type& lhs, const cType<Type>& rhs){
cType<Type> c_1(lhs,0), c_2=rhs;
if(c_1.r == c_2.r && c_1.i == c_2.i) {return true;}
else{return false;}
}
template<class Type>
bool operator !=(const cType<Type>& lhs, const cType<Type>& rhs){
cType<Type> c_1=lhs, c_2 = rhs;
return !(c_1==c_2);
}
template<class Type>
bool operator !=(const cType<Type>& lhs, const Type& rhs){
cType<Type> c_1=lhs, c_2(rhs,0);
return !(c_1==c_2);
}
template<class Type>
bool operator !=(const Type& lhs, const cType<Type>& rhs){
cType<Type> c_1(lhs,0), c_2=rhs;
return !(c_1==c_2);
}

// Standard functions
template<class Type>
Type abs(const cType<Type>& z){
cType<Type> c = z;
Type abs = sqrt(c.r*c.r + c.i*c.i);
return abs;
}
template<class Type>
Type arg(const cType<Type>& z){ // Returns Arg(z)
cType<Type> c = z;
return atan2(c.i,c.r);
}
template<class Type>
cType<Type> conj(const cType<Type>& z) {
cType<Type> c = z;
c.i = - c.i;
return c;
}

// Exponential functions
template<class Type>
cType<Type> exp(const cType<Type>& z) {
cType<Type> c = z;
Type temp = exp(c.r)*cos(c.i);
c.i = exp(c.r)*sin(c.i);
c.r = temp;
return c;
}
template<class Type>
cType<Type> log(const cType<Type>& z){
cType<Type> c = z;
Type r = abs(z), theta = arg(z);
c.r = log(r);
c.i = theta;
return c;
}

// Power functions
template<class Type>
cType<Type> pow(const cType<Type>& z1, const cType<Type>& z2){
cType<Type> c1=z1, c2=z2, c3;
Type a=c1.r,b=c1.i, c=c2.r,d=c2.i;
c3.r = pow((a*a+b*b),(c/2))*exp(-d*arg(c1))*
(cos(c*arg(c1)+0.5*d*log(a*a+b*b)));
c3.i = pow((a*a+b*b),(c/2))*exp(-d*arg(c1))*
(sin(c*arg(c1)+0.5*d*log(a*a+b*b)));
return c3;
}
template<class Type>
cType<Type> pow(const cType<Type>& z, const Type& t){
cType<Type> c1=z, c2(t,0);
c1 = pow(c1,c2);
return c1;
}
template<class Type>
cType<Type> pow(const Type& t, const cType<Type>& z){
cType<Type> c1=z, c2(t,0);
c1 = pow(c2,c1);
return c1;
}
template<class Type>
cType<Type> sqrt(const cType<Type>& z){
cType<Type> c1=z, c2(0.5,0);
c1 = pow(c1,c2);
return c1;
}

// Trigonometric functions
template<class Type>
cType<Type> sin(const cType<Type>& z){
cType<Type> c = z, i(0,1);
c = (exp(i*c) - exp((-(Type)1)*i*c)) / ((Type)2 * i);
return c;
}
template<class Type>
cType<Type> cos(const cType<Type>& z){
cType<Type> c = z, i(0,1);
c = (exp(i*c) + exp(c/i)) / ((Type)2);
return c;
}
template<class Type>
cType<Type> tan(const cType<Type>& z){
cType<Type> c = z;
c = sin(c) / cos(c);
return c;
}
template<class Type>
cType<Type> asin(const cType<Type>& z){
cType<Type> c = z, i(0,1);
c = (-(Type)1)*i*log( i*c + sqrt((Type)1 - (c*c)) );
return c;
}
template<class Type>
cType<Type> acos(const cType<Type>& z){
cType<Type> c = z;
c = asin((-(Type)1 ) * c) + (Type)(M_PI/2);
return c;
}
template<class Type>
cType<Type> atan(const cType<Type>& z){
cType<Type> c = z, i(0,1);
c = (-(Type)(0.5))*i*(log((Type)1 - (i*c)) - log((Type)1 + (i*c) ) );
return c;
}
template<class Type>
cType<Type> sinh(const cType<Type>& z){
cType<Type> c = z, i(0,1);
c = sin((-(Type)1)*i*c) * i;
return c;
}
template<class Type>
cType<Type> cosh(const cType<Type>& z){
cType<Type> c = z, i(0,1);
c = cos(c/i);
return c;
}
template<class Type>
cType<Type> tanh(const cType<Type>& z){
cType<Type> c = z;
c = sinh(c) / cosh(c);
return c;
}
	\end{lstlisting}
	
\end{frame}