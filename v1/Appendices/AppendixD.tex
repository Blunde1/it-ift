\chapter{Volatility Swaps and the VIX}
\label{AppendixD}
\lhead{Appendix D. \emph{Volatility Swaps and the VIX}}

We here give the full derivation of the formula for the VIX.
The reason for doing this is that the VIX is often dumbed down, or explained in superficial ways.
The derivation of the formula for a volatility swap can be found in \cite{carr1998towards},
the relation to the VIX volatility indices can be found in \cite{carr2005tale},
and the transformation to yearly expected volatility in percentages together with practical implementation can be found in \cite{exchange2009cboe}.

The VIX indices for the S\&p500 are calculated in the following way:
\begin{align}
	\text{VIX} = 100 K_{\text{var}},
\end{align}
where $K_{\text{var}}$ is approximated expected volatility of the S\&P500.
It is calculated in the following way:
\begin{align}
	K_{var}\approx\frac{2}{T}e^{\mu T}\sum_{i=1}^{n}\frac{\Delta K_i}{K_i^2}Q(K_i)-\frac{1}{2}\left(\frac{F_0}{S^*}-1\right)^2,
\end{align}
where 


The vix is for many a blackbox.

\section{Variance Swap}
At maturity, the payoff of a variance swap is
\begin{align*}
10000N(\Sigma^2-K_{var})
\end{align*}
where 
\begin{enumerate}
	\item $N$ = notational amount quoted in \$ per volatility point squared (hence the factor of $100^2=10000$).
	\item $\Sigma^2$ = realized variance in the underlying asset during the life of the swap.
	\item $K_{var}$ = strike on variance.
\end{enumerate}

The underlying assumption is that the S\&P500, $S_t$, follows a geometric brownian motion

\begin{align*}
dS_t&=\mu_tS_tdt+\sigma_tS_tdB_t.
\end{align*}
($\mu$ is usually set to be the short rate $r$)

This is probably the most famous stochastic differential equation which we solve using Itô's lemma (this is probably trivial to you, but I don't know how much you know so please don't be offended that I'm showing this).\\

Use the substitution $Y_t=\log (S_t)$ and apply It\^o's Lemma to the function $f(t,x):=\log (x)$.
The partial derivatives are

\begin{align}
\frac{\partial f}{\partial t}(t,X_t)=0,\;
\frac{\partial f}{\partial x}(t,X_t)=\frac{1}{S_t},\;
\frac{\partial^2 f}{\partial^2 x}(t,X_t)=-\frac{1}{S_t^2}. 
\end{align}
Thus we obtain 
\begin{align*}
dY_t&=\left(\frac{\mu_tS_t}{S_t}-\frac{\sigma_t^2S_t^2}{2S_t^2}\right)dt+
\frac{\sigma_tS_t}{S_t}dB_t \text{ by It\^o's Lemma} \\
&=\left(\mu_t-\frac{\sigma_t^2}{2}\right)dt+\sigma_tdB_t
\end{align*}

Dividing the equation for $dS_t$ with $S_t$ and subtracting this last equation we obtain:

\begin{align*}
\frac{dS_t}{S_t}-d(\log(S_t))&=\mu_tdt+\sigma_tdB_t-\left(\left(\mu_t-\frac{\sigma_t^2}{2}\right)dt+\sigma_tdB_t\right)\\
&=\frac{\sigma_t^2}{2}dt.
\end{align*}
Hence,
\begin{align*}
Variance = \frac{1}{T}\int_{0}^{T}\sigma_t^2dt=\frac{2}{T}\left(\int_{0}^{T}\frac{dS_t}{S_t}-\log(\frac{S_T}{S_0})\right).
\end{align*}

Notice that
\begin{align*}
\log\left(\frac{S_T}{S_0}\right)=\log\left(\frac{S_T}{S^*}\right)+\log\left(\frac{S^*}{S_0}\right)
\end{align*}

We now make use of the dirac delta function ($\delta(x)=\infty$ if $x=0$, $0$ otherwise) to write on $-\log\left(\frac{S_T}{S_*}\right)$ and then integrate by parts twice (this is where the $1/K^2$ comes from):

\begin{align*}
-\log\left(\frac{S_T}{S_*}\right)&=\log (S^*)-\log (S_T)\\
&=\log (S^*)-\int_{o}^{\infty}log(K)\delta(S_T-K)dK\text{ by propertis of the dirac delta function}\\
&=\log (S^*)-\int_{o}^{S^*}log(K)\delta(S_T-K)dK-\int_{S^*}^{\infty}log(K)\delta(S_T-K)dK \text{ splitting the integral}\\
&=\log (S^*)-\left[\log(K)1_{(S_T<K)}\right]_0^{S^*}+\int_{0}^{S^*}\frac{1}{K}1_{(S_T<K)}dK\\
&-\left[\log(K)1_{(S_T\geq K)}\right]_{S^*}^{\infty}+\int_{S^*}^{\infty}\frac{1}{K}1_{(S_T\geq K)}dK\text{ by integration by parts}\\
&=\log (S^*)- \log(S^*)1_{(S_T<K)}+\left.\frac{1}{K}[K-S_T]^+\right|_0^{S^*}+\int_{0}^{S^*}\frac{1}{K^2}[K-S_T]^+dK\\
&-\log(S^*)1_{(S_T\geq K)}+\left.\frac{1}{K}[S_T-K]^+\right|_{S^*}^\infty+\int_{S^*}^{\infty}\frac{1}{K^2}[S_T-K]^+dK\\
&=\frac{1}{S^*}\left(\left[S^*-S_T\right]^+-\left[S_T-S^*\right]^+\right)
+\int_{0}^{S^*}\frac{1}{K^2}[K-S_T]^+dK+\int_{S^*}^{\infty}\frac{1}{K^2}[S_T-K]^+dK\\
&=\frac{1}{S^*}\left(S^*-S_T\right)
+\int_{0}^{S^*}\frac{1}{K^2}[K-S_T]^+dK+\int_{S^*}^{\infty}\frac{1}{K^2}[S_T-K]^+dK
\end{align*}

This was kinda tedious, but this calculation is as noted before the reason for $1/K^2$ part in the final equation.\\

We now have the final equation for the variance

\begin{align*}
Variance=\frac{2}{T}\left(\int_{0}^{T}\frac{dS_t}{S_t}-\log(\frac{S^*}{S_0})
+\frac{1}{S^*}\left(S^*-S_T\right)
+\int_{0}^{S^*}\frac{1}{K^2}[K-S_T]^+dK+\int_{S^*}^{\infty}\frac{1}{K^2}[S_T-K]^+dK
\right)
\end{align*}

Just realizing I should have said that $Variance = \Sigma^2$ (realized variance in the underlying asset during the life of the swap). To avoid arbitrage in the payoff equation (equation located in the start) we have to set:
\begin{align*}
E[\Sigma^2]=K_{var},
\end{align*}
the expectation of the realized variance is equal to the strike on the variance.

For the expecattions, not that
\begin{align*}
E\left[\int_{0}^{T}\frac{dS_t}{S_t}\right]&=E\left[\int_{0}^{T}\mu_tdt+\int_{0}^{T}\sigma_tdB_t\right]=\mu T \text{ assuming }\mu\text{ is constant}\\
E\left[\log(\frac{S^*}{S_0})\right]&=\log(\frac{S^*}{S_0})\text{ they're just constants}\\
E\left[S_T\right]&=S_0e^{\mu T}\\
\end{align*}

Also observe that the price of a put and a call option both with strike $K$ is
\begin{align*}
P(K)&=e^{-eT}E\left[\left(K-S_T\right)^+\right]\\
C(K)&=e^{-eT}E\left[\left(S_T-K\right)^+\right]\\
\end{align*}

Taking expectations of the equation for the realized variance we thus obtain 

\begin{align*}
K_{var}=E[Variance]=\frac{2}{T}\left(\mu T-\left(\frac{S_O}{S^*}e^{\mu T}-1\right)
-\log\left(\frac{S^*}{S_0}\right)
+e^{\mu T}
\int_{0}^{S^*}\frac{P(K)}{K^2}dK+e^{\mu T}\int_{S^*}^{\infty}\frac{C(K)}{K^2}dK
\right)
\end{align*}
Another way of writing this is:
\begin{align*}
K_{var}&=\frac{2}{T}\left(\log\left(\frac{F_o}{S^*}\right)-\left(\frac{F_0}{S^*}-1\right)+e^{\mu T}
\int_{0}^{S^*}\frac{P(K)}{K^2}dK+e^{\mu T}\int_{S^*}^{\infty}\frac{C(K)}{K^2}dK
\right)
\end{align*}
Now, making the smart choice of setting $S^*=F_0=S_0e^{\mu T}$ a lot cancel out and we have the final equation

\begin{align*}
K_{var}=\frac{2e^{\mu T}}{T}\left[\int_{0}^{F_0}\frac{P(K)}{K^2}dK+\int_{F_0}^{\infty}\frac{C(K)}{K^2}dK\right].
\end{align*}



\section{Relationship between variance swap and the VIX}
cite the S\&P paper

We now want to make a discretized version (the VIX) of our fair price variance swap formula. Assume that the price of options with strike prices $K_i,\,(i=1:n)$ are known and that
\begin{align*}
K_1<K_2<...<K_n
\end{align*}
abd choose $S^*$ equal to the first strike price below $F_0$, define the function

\begin{align*}
Q(K_i)&=P(K_i)1_{(K_i\leq S^*)}+C(K_i)1_{(S^*<K_i)},
\end{align*} 

then approximate the integrals as following

\begin{align*}
e^{\mu T}
\int_{0}^{S^*}\frac{P(K)}{K^2}dK+e^{\mu T}\int_{S^*}^{\infty}\frac{C(K)}{K^2}dK=e^{\mu T}\sum_{i=1}^{n}\frac{\Delta K_i}{K_i^2}Q(K_i)
\end{align*}

where
\begin{align*}
\Delta K_i&=\frac{K_{i+1}-K_{i}}{2},\,\,i=2:n-1\\
\Delta K_1&=K_2-K_1\\
\Delta K_n&=K_n-K_{n-1}.
\end{align*}

Set this aproximation inside the equation for fair variance swap with and we obtain
\begin{align*}
K_{var}\approx\frac{2}{T}\left(\log\left(\frac{F_o}{S^*}\right)-\left(\frac{F_0}{S^*}-1\right)+e^{\mu T}\sum_{i=1}^{n}\frac{\Delta K_i}{K_i^2}Q(K_i)\right)
\end{align*}

The maclaurin polynomial for $\log(F_0/S^*)$ is
\begin{align*}
\log\left(\frac{F_0}{S^*}\right)&=\left(\frac{F_0}{S^*}-1\right)-\frac{1}{2}\left(\frac{F_0}{S^*}-1\right)^2+O\left(\left(\frac{F_0}{S^*}-1\right)^2\right)
\end{align*}

Thus

\begin{align*}
\log\left(\frac{F_o}{S^*}\right)-\left(\frac{F_0}{S^*}-1\right)
=-\frac{1}{2}\left(\frac{F_0}{S^*}-1\right)^2+O\left(\left(\frac{F_0}{S^*}-1\right)^2\right),
\end{align*}
and we obtain our final approximation

\begin{align*}
K_{var}\approx\frac{2}{T}e^{\mu T}\sum_{i=1}^{n}\frac{\Delta K_i}{K_i^2}Q(K_i)-\frac{1}{2}\left(\frac{F_0}{S^*}-1\right)^2
\end{align*}
which we recognize as the fomula for the VIX.

\hrule
\mbox{}
\vspace{1cm}