% Appendix A

\section{Multiple Itô Integrals}
\label{AppendixA}
%\lhead{Appendix A. \emph{Itô Integrals}}

The evaluations of the integrals \eqref{Chap4.1 ito integrals} involved in the development of the discretization schemes in section \ref{Chap4.1} were presented without proofs.
For the non-trivial ones, we here show how they can be calculated.


The first integral of concern is the integral $I_{1,0}$.
The calculation involves using the Fubini theorem for stochastic integrals (see e.g. \citet[ p.~477]{bjork2009arbitrage}) for switching the order of integration:
\begin{align}\label{App: I1}
I_{1,0}&=\int_0^t \int_0^{s_1}dW_{s_2}ds_1
=\int_0^t W_{s_1}ds_1\notag\\
&=\int_0^t \int_{0}^{t} \mathbbm{1}_{[0,s_1]}(s_2)dW_{s_2}ds_1%\notag\\
=\int_0^t \int_0^t \mathbbm{1}_{[0,s_1]}(s_2)ds_1dW_{s_2}\notag\\
&=\int_0^t t-s_2dW_{s_2}%\notag\\
\sim N\left(0,\int_0^t(t-s_2)^2ds_2 \right)\notag\\
&\sim N\left( 0,\frac{1}{3}t^3 \right).
\end{align}
The second integral of interest is the integral $I_{0,1}$.
We here wish to show that the equation $I_{0,1}=tJ_1-J_2$, where $J_1$ and $J_2$ are as defined in section \ref{Chap4.1}, is valid.
Define $Y$ by $Y_t=tW_t$.
Then we have $Y_t = f\left(t,X_t\right)$, where $f(t,x)=tx$, $X_t=W_t$, and $Y_0=0$.
The partial derivatives are $\frac{\partial f}{\partial t}=x$, $\frac{\partial f}{\partial x}=t$, and $\frac{\partial^2 f}{\partial^2 x}=0$.
We also trivially have $dX = dW$.
Itô's lemma then gives
\begin{align}
dY=W_tdt + tdW_t,
\end{align}
which in integral form yields
\begin{align}
tW_t=Y_t&=\int_0^tW_sds + \int_0^t sdW_s=W_t + \int_{0}^{t}sdW_s.
\end{align}
From this we see that the equation holds.

The third and final integral, $I_{1,1}$, is commonly used as an example to illustrate the Itô integral and can be found in most textbooks on the subject.
It can of course be computed directly from the definition, but also via an application of Itô's lemma, similar to that of $I_{0,1}$.
We first calculate the inner integral, and then we follow \cite{bjork2009arbitrage} and the application of Itô's lemma found there:
\begin{align}
I_{1,1}&=\int_0^t \int_0^{s_1} dW_{s_2}dW_{s_1}= \int_0^t W_{s_1}dW_{s_1}.
\end{align}
Define $Y_t=W_t^2$, then $Y_0=0$ and $Y$ can be written as $Y_t=f(t,X_t)$, where $X_t=W_t$ and $f$ is a function such that $f(t,x)=x^2$.
The partial derivatives of $f$ are $\frac{\partial f}{\partial t}=0$, $\frac{\partial f}{\partial x}=2x$, and $\frac{\partial^2 f}{\partial^2 x}=2$.
From Itô's lemma we then have
\begin{align}
dY_t&=2XdX + \frac{1}{2}2(dX)^2=dt + 2W_tdW_t,
\end{align}
since $dX=dW$.
In integral form this reads
\begin{align}
W_t^2=Y_t=t + 2\int_0^t W_sdW_s,
\end{align}
which trivially implies that $I_{1,1}=\frac{1}{2}\left( J_1^2 - t \right)$.

Our final consideration is the covariance between $J_1$ and $J_2$,
\begin{align}
Cov(J_1,J_2)&=Cov\left( W_t, \int_0^t W_s ds \right)=Cov\left( \int_0^tdW_s,\int_0^t \int_0^{s_1} dW_{s_2}ds_1  \right).
\end{align}
Applying the Fubini theorem as for $I_{1,0}$ \eqref{App: I1}, and by the properties of the Itô integral \eqref{Chap2.1 Ito integral}, we obtain
\begin{align}
Cov(J_1,J_2)&=Cov\left( \int_0^tdW_s,\int_0^t (t-s)dW_s  \right)=\int_0^t 1*(t-s)ds=\frac{1}{2}t^2.
\end{align}